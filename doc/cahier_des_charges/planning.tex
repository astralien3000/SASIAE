\begin{tabular}{rp{8cm}|p{3cm}}

    2 décembre 2013 &&\\
    
    %2
    &
    La fenêtre principale de l'application &
    Nicolas, Théotime \\
    
    &&\\
    
    %2
    &
    Première version d'Aversive++ minimale &
    Loïc, Clément \\
    
    &&\\
    
    %2
    &
    Environnement physique minimal &
    Jean-Raymond, Hugo \\
    
    &&\\
    
    %1
    &
    Les modules encodeur et moteur &
    Benoît \\
    
    23 décembre 2013 &&\\
    
    %3
    &
    Intéraction entre le code robot et modules &
    Clément, Benoît, Hugo \\
    
    &&\\
    
    %2
    &
    Modules GP2, AX12, Servomoteur, Balise de détection &
    Loïc, Théotime \\
    
    &&\\
    
    %2
    &
    Complétion d'Aversive++ avec les interfaces GP2, AX12, Servomoteur, Balise de détection &
    Jean-Raymond, Nicolas\\

    27 javier 2014 &&\\

    %3
    &
    Intégration du moteur physique au coordinateur &
    Loïc, Nicolas, Jean-Raymond \\
    
    &&\\
    
    %2
    &
    Introduction des erreurs &
    Théotime, Hugo \\
    
    &&\\
    
    %2
    &
    L'environnement physique complet &
    Clément, Benôit \\
    
    10 février 2014 &&\\
    
    %2
    &
    Charger dynamiquement les modules en mémoire &
    Théotime, Benoît \\
    
    &&\\
    
    %2
    &
    Afficher l'environnement sur la fenêtre principale &
    Loïc, Hugo \\
    
    &&\\
    
    %1
    &
    Charger des fichiers de configurations (gui) &
    Jean-Raymond \\
    
    &&\\
    
    %2
    &
    Le coordinateur permettant plusieurs robots &
    Clément, Nicolas \\
    
    3 mars 2014 &&\\
    
    %3
    &
    Pouvoir placer les robots sur la table &
    Théotime, Benoït, Hugo \\
    
    &&\\
    
    %4 (2 aversive + 2 modules)
    &
    UART, I2C, Lidar (Aversive++ et modules) &
    Loïc, Clément, Nicolas, Jean-Raymond \\
    
    24 mars 2014 &&\\
    
    &
    Rendu final &
    \\
    
\end{tabular}
