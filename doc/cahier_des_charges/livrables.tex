Ici sont listés les différents livrables à produire en cours de projet :

%\begin{enumerate}
    \subsubsection{Première série de livrables}
    \begin{itemize}
    
        \item{La fenêtre principale de l'application, décrite plus haut, et dont une image a été montrée.}
        
        \item{Une première version d'Aversive++ minimale permettant de coder avec les actionneurs et capteurs suivants : Encodeur, Moteurs de propulsion, gestion des tâches.\\ Les interfaces implémenteront la communication avec le simulateur et des tests permettront de vérifier cette communication.}

        \item{Un environnement physique minimal gérant un robot (codé en dur) se déplaçant sur une table plane et permettant de modifier la vitesse de ses moteurs.}
        
        \item{Les modules encodeur et moteur.}
        
    \end{itemize}
    \subsubsection{Deuxième série de livrables}
    \begin{itemize}
        
        \item{Le coordinateur minimal permettant l'interaction entre le code robot et modules.}
        
        \item{Complétion de l'API avec : GP2, AX12, Servomoteur, Balise de détection, gestion des overflows.}
        
        \item{Ajout des modules GP2, AX12, Servomoteur, Balise de détection.}
        
    \end{itemize}
    \subsubsection{Troisième série de livrables}
    \begin{itemize}
        
        \item{L'intégration du moteur physique au coordinateur.}
        
        \item{Introduction des erreurs dans le retour de capteurs et dans l'execution des commandes par les actionneurs.}
        
        \item{L'environnement physique complet, avec plusieurs robots, et une table chargeable.}
        
    \end{itemize}
    \subsubsection{Quatrième série de livrables}
    \begin{itemize}
        
        \item{Complétion du coordinateur pour qu'il puisse charger dynamiquement les modules en mémoire et rediriger les ordres vers les modules adéquats.}
        
        \item{Afficher l'environnement de la table avec les robots sur la fenêtre principale, ainsi que la valeur des capteurs et le log de tous les évènements.}
        
        \item{La partie de l'interface graphique permettant de charger des fichiers de configurations.}
        
        \item{Le coordinateur permettant plusieurs robots dans le scénario. Il doit aussi permettre de charger les robots, la table depuis un fichier, et uniquement les modules utilisés par les robots.}
        
    \end{itemize}
    \subsubsection{Derniers livrables}
    \begin{itemize}
        
        \item{Pouvoir placer les robots sur la table en début de simulation depuis l'interface graphique.}
        
        \item{Les interfaces de l'API, et les modules de simulateur associés pour tous les actionneurs/capteurs suivants : UART, I2C, Lidar.}
        
        \item{Documentation du projet:}
			\begin{itemize}
				\item Une documentation de l'API avec exemples de code.
				\item Une documentation pour la création de nouveau modules.
				\item Des tutoriels d'utilisation du simulateur.
			\end{itemize}
        
    \end{itemize}
%\end{enumerate}
