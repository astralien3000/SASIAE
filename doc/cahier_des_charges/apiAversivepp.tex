\subsection{Besoins fonctionnels}

Eirbot, l'association cliente, développe l'API 2.0 \texttt{Aversive++} à partir de l'API Aversive. Néanmoins cette API possèdera au moins deux implémentations différentes que nous noterons implémentation\_AVR et implémentation\_simulation. Une qui sera réalisée par Eirbot et qui doit être compilée avec le code robot afin de faire fonctionner celui-ci dans des conditions réelles. 

L'autre implémentation est un des livrables de SASIAE. Il faut noter que certains éléments de implémentation\_AVR ne seront pas simulés par implémentation\_simulation, mais cette dernière reprendra l'implémentation de l'implémentation\_AVR. Nous aurons donc une partie commune aux deux : implémentation\_Commune. 
Concrètement dans un premier temps seuls les devices seront simulés. Donc seulement cette partie de l'API sera implémentée de manière "complète" dans implémentation\_simulation. Nous pouvons cependant noter que l'interface d'\texttt{Aversive++} sera %presque (Clément) %
identique quelque soit l'implémentation utilisée.

\subsubsection{Gestion des entrées sorties du robot}

Le robot est composé de différents capteurs et actionneurs, le but de l'API est de proposer des interfaces pour intéragir avec ces éléments, en permettant de ne pas avoir à se soucier de l'architecture matérielle. La gestion se fait de la manière suivante : à chaque capteurs/actionneurs correspond une classe dans l'API. Les classes qui devront être présentes dans le livrable sont celles qui assurent la communication avec le capteur GP2, l'AX12, le moteur à courant-continu, le servo-moteur et le moteur propulsion.
Soit 5 classes.

\subsubsection{Gestion des modules du microprocesseur}

Tous les microprocesseurs proposent plus ou moins les mêmes services. Il faut pouvoir les configurer sans avoir à connaître les registres à modifier. Cependant cette partie ne sera utilisable que sur un robot, et pas avec le simulateur. En effet, le simulateur n'a pas besoin de fournir une interface de si bas niveau pour tester l'asservissement et la stratégie. Les interfaces seront présentes dans Aversive++ mais ne feront pas grand chose dans le cas où le code est compilé pour la simulation.

%% TODO
%% Hum !!! Il faut tout de même les simuler (ou tout du moins une partie) sinon, aurevoir le scheduler (Clément)
%% On peut directement coder le scheduler sans coder un timer. (Loïc)

\subsubsection{Détecter les erreurs d'opération}

Le manque de mémoire et de puissance de calcul sur les AVR utilisés à Eirbot nous poussent à contrôler et réduire la taille des entiers utilisés. Cela augmente donc les chances de produire des overflows lors des opérations, qui sont difficilement détectables sur une architecture PC de 32 ou 64 bits. En effet, les registres du processeur étant plus grand, si la suite d'opération ammenant l'overflow aboutit à un résultat compris dans les limites du type de départ, l'overflow n'aura pas lieu sur un PC.

Ces overflows ont de graves conséquences dans le comportement du robot. De ce fait, il est important d'avoir un mécanisme de détection de ces erreurs d'opérations arithmétiques lors de la simultation.

La solution retenue est de passer par des classes qui ré-implémentent les types primitifs utilisés afin d'en surcharger les opérations. De cette manière toutes les étapes du calcul sont maitrisées par l'API.

Dans le cas d'une erreur, un message sera envoyé au simulateur pour en avertir l'utilisateur.

\subsubsection{Planification des tâches}

Sur les robots, nous avons besoin de pouvoir exécuter des routines à intervalle régulier (une routine pour détecter la présence d'un obstacle afin d'arrêter les moteurs par exemple). C'est pourquoi nous avons besoin d'un scheduler de tâches. Celui ci devra nous permettre :
\begin{itemize}
    \item d'ajouter une nouvelle routine avec l'intervalle de temps entre chaque appel ainsi que sa priorité ;
    \item de supprimer une routine du scheduler ;
\end{itemize}

\subsection{Besoins non fonctionnels}

\subsubsection{Langage de programmation}

Le langage de programmation choisi est le C++. C'est celui qui répond le mieux à nos attentes. En effet l'API est pensée à l'aide des mécanismes objet. Le C++ est un langage orienté objet qui s'adapte donc plus que le C au développement de l'API, notemment pour la surcharge des opérateurs pour détecter les overflows. De plus nous avons besoin d'un code qui puisse s'adapter aux différentes architectures de micro-contrôleur et afin de posséder cette généricité nous pouvons, avec le langage C++, utiliser le mécanisme des \og templates \fg\ qui est adapté. Enfin nous écartons le JAVA, car même s'il est orienté objet il est souhaitable de garder la compatibilité avec le code Robot qui est en langage C.

\subsubsection{Compatibilité AVR - x86}

La même API est utilisée pour la simulation sur PC et pour le robot. Il est proposé de la séparer en 3 parties principales :
\begin{itemize}
    \item{La partie indépendante de l'architecture.}
    \item{La partie dépendante de l'architecture, côté simulation.}
    \item{La partie dépendante de l'architecture, côté AVR.}
\end{itemize}

%% Clément - peut faire redondance avec le premier paragraphe de cette subsection. A débattre
La partie AVR ne rentre pas ici dans le cadre du projet mais l'interface de l'API étant unique et indépendante de la plateforme finale sur laquelle s'exécute le code l'utilisant, l'interface devra respecter les contraintes imposées par les AVR (voir partie suivante entre autres).

\subsubsection{Taille des executables}

L'architecture actuellement utilisée n'offre dans le meilleur des cas que 128 Ko de mémoire flash destinée à stocker l'exécutable, et 4 Ko de RAM. Il faut donc être capable de générer un code le plus léger possible.

Il faut donc proposer une interface qui permette d'utiliser des techniques visant à réduire la taille de l'éxécutable avec l'optimisation proposée par le compilateur G++, qui sera le compilateur utilisé principalement par Eirbot. Ainsi, on peut énoncer quelques règles de conception :
\begin{itemize}
    \item{Pas de fonction virtuelle.}
    \item{Utilisation des variables constantes dans les cas le permettant.}
    \item{Utilisation des fonctions inline pour les "petites fonction" (3 ou 4 instructions maximum).}
\end{itemize}

La simulation se faisant sur un ordinateur, l'implémentation implémentation\_Simulation ne sera pas contrainte par les considérations de taille. Cependant les déclarations de classes devront suivre ces rêgles.


