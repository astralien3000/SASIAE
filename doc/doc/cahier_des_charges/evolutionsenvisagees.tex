%%\begin{itemize}

%%% (ldauphin) : je les ait placé dans validation : voir le main.tex pour explication de la raison + cours de Génie Logiciel
%%% Je les garde au cas ou j'ai oublié quelque chose

%%\item{Premier incrément :}
%%GUI: affiche 2D de la scène, la valeur des capteurs et des messages de log et démarage de la simulation.
%%Moteur physique : interaction avec le coordinateur et le GUI.
%%Coordinateur : table et robot codé en dur, communication avec le robot et les modules.
%%modules API: timer, gp2, moteur et roue codeuse.
%%modules simulation: timer, gp2, moteur et roue codeuse.

%%\item{Deuxieme incrément :}
%%les autres modules (uart, i2c, servomoteur, ax12), mode pause/resume dans la simulation, ajout d'obstacle.

%%\item{Troisième incrément :}
%%Charger les donnés de la table et du robot (plus codé en dur), erreur des capteurs.

%%\item{Dernier incrément :}
%%Prendre en compte l'ensemble des pièces et leurs caractéristiques (matérieux, masse, coeficient d'adérence), tous les actionneurs du robot %%dans la simulation (via le moteur physique) et leurs interactions avec l'environement (servomoteur, moteur pas à pas ...).
%%\end{itemize}

Liste des améliorations à apporter au projet :
\begin{itemize}
	\item{Ajout d'un éditeur de table.}
	\item{Ajout d'un éditeur de robot.}
	\item{Avoir l'affichage de la simulation en 3D.}
	\item{Mise en place des règles de jeux avec calcul des scores.}		
	\item{Ajout d'un fonctionnement temps réel avec le robot branché (type Hardware-in-the-loop) incluant le réglage automatique des paramètres d'asservissement.}
	\item{Génération de code à partir de l'interface graphique.}
\end{itemize}

