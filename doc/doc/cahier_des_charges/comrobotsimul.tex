%%benoit :
\section{Communication entre l'exécutable robot et le logiciel de simulation}

Le code du robot doit être un exécutable indépendant du logiciel de simulation, cet exécutable robot doit être produit par une simple compilation en utilisant l'implementation simulateur d'Aversive++. Finalement, ces deux exécutables, celui du code robot et celui du logiciel, doivent communiquer ensemble.

\subsection{Moyen de communication}

%Hugo
La communication entre le logiciel de simulation et l'exécutable robot
doit être bidirectionnelle, nous utiliserons donc des socket au travers de la classe QTcpSocket fournit par le framework Qt.
Cela permet à l'exécutable robot d'envoyer
des ordres aux actionneurs, et des messages à l'utilisateur. Ou dans
l'autre sens, la simulation peut transmettre la valeur des différents
capteurs simulés à l'exécutable robot, tout comme elle peut simuler un module
interne au microcontrôleur (Horloge, UART, Analogic to Digital
Converter, ports). Ces modules dépendent fortement du micro-contrôleur
utilisé. Ces messages ne sont pas synchrone et
n'attendent aucune réponse.
%HUGO : Vérifier qu'on parle bien qlqpart dans le cahier des charges 
% du fait qu'on importe en bibliothèque dynamique tout les modules
% nécessaire, c'est-à-dire leur émulation.
% Clément: bien précisé dans bibliotheque.tex je pense ;)

Pour que les deux exécutables puissent identifier les objets auxquels
ils parlent, ils utilisent le port utilisé dans le cas d'une ressource
externe au microcontrôleur (branchée sur une/des pins de la carte donc), ou bien le nom de cette ressource si elle est interne au microcontrôleur. % interne à quoi ? 
Dans le cas où l'exécutable du robot envoie sur un port des
données ne correspondant pas aux valeurs attendu par le coordinateur,
un message d'erreur est envoyé à l'utilisateur.


Pour faire la communication, on utilise un socket dans lequels on envoie des messages codé en ASCII séparés par des sauts de lignes UNIX (caractère 10).
%quel format de fichier pour ces messages ? 
\subsection{Syncronisation simulation - exécution robot}

Pour indiquer au code robot qu'il peut faire des calculs, c'est à dire
qu'un pas de simulation est à faire coté robot, un message de
synchronisation est envoyé.

%% Clément : wut? je veux dire, on simule simplement le module timer du microcontrolleur que le scheduler utilisera et ça devrait être aussi simple que cela puisque le robot devrait être bloqué dans un while(1) le reste du temps
%% (Loïc) Sauf que le wile(1) sera pas vide. Du moins pas forcément.

\subsection{Detail des messages échangeables}
Les messages doivent contenir les champs suivant :
\begin{itemize}
    \item{Type de message :}
    \begin{itemize}
        \item{Texte pour l'utilisateur}
        \item{Message d'un module}
        \item{Valeur d'un capteur / actionneur}
        \item{Synchronisation}
    \end{itemize}
\end{itemize}

Dans le cas des messages texte à l'intention de l'utilisateur :
\begin{itemize}
    \item{Le type de message :}
    \begin{itemize}
        \item{Info}
        \item{Warning}
        \item{Error}
    \end{itemize}
    \item{Le message texte :}
    \begin{itemize}
        \item{Le texte en une seule ligne}
    \end{itemize}
\end{itemize}

Dans le cas des valeurs d'un objet :
\begin{itemize}
    \item{Identificateur de l'objet :}
    \begin{itemize}
        \item{pins auquelles il est branché.}
    \end{itemize}
    \item{Types des valeurs : (peut etre un tableau à deux dimension de valeurs)}
    \begin{itemize}
        \item{Nombre de lignes}
        \item{Type de la première colonne (parmi bool, int, float, charactère)}
        \item{Type de la deuxième colonne}
        \item{...}
    \end{itemize}
    \item{Valeurs}
    \begin{itemize}
        \item{Séparées par des virgules}
        \item{Avec les chaines de caractères entre double-quotes}
    \end{itemize}
\end{itemize}

Message de module :
\begin{itemize}
    \item{Type de l'objet}
    \begin{itemize}
        \item{doit correspondre au nom d'un module dynamique}
    \end{itemize}
    \item{Identificateur de l'objet}
    \begin{itemize}
        \item{pins dans le cas d'un module externe au microcontroleur}
        \item{nom de la ressource pour les modules internes au microcontroleur}
    \end{itemize}
    \item{Chaine de caractères à destination du module dynamique gérant le périphérique}
\end{itemize}

Pour les initialisations, c'est un message de module contenant la chaine de caractère INIT, suivit des paramètres séparé par des espaces.

Pour les messages de syncronisation deux type de message :
\begin{itemize}
    \item{Nouveau pas de simulations}
    \item{Code robot en attente}
\end{itemize}
