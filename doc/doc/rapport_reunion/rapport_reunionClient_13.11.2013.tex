\documentclass[a4paper,10pt]{article}

% Options possibles : 10pt, 11pt, 12pt (taille de la fonte)
%                     oneside, twoside (recto simple, recto-verso)
%                     draft, final (stade de développement)

\usepackage[utf8]{inputenc}   % pour les accents
\usepackage[T1]{fontenc}      % Police contenant les caractères français
\usepackage[francais]{babel}  % Placez ici une liste de langues, la
                              % dernière étant la langue principale

\usepackage[a4paper]{geometry}% Réduire les marges
% \pagestyle{headings}        % Pour mettre des entêtes avec les titres
                              % des sections en haut de page

\title{Rapport de réunion}           % Les paramètres du titre : titre, auteur, date
\author{SASIAE}
\date{17 novembre 2013}

\sloppy                      % pour que les lignes ne débordent pas dans les marges

\begin{document}

\maketitle                    % Faire un titre utilisant les données
                              % passées à  \title, \author et \date

\begin{abstract}
Compte rendu de la réunion du mercredi 13 novembre 2013 avec le client Aymeric Vincent.
\end{abstract}
Problème représentation: lecture difficile
Que veut dire gérer les com & fils

Prendre des exemples:
Sur un robot, que veut dire flux

Le découpage doit permettre d'éclaircir son idée

Au final les fonctions devront être plus puissant

Utilisation de org-mode sour Emacs --> suite d'item en html ou latex

Problème de la gestion des opérations... Instrumentation de code, >> surcharge des opérateurs << (le c++ peut pas le faire a priori)

Le bloc sûreté de fonctionnement, intéressant, recherche là-dessus, est-ce vraiment nécessaire? Si oui, solution

En fait c'est quoi ce document?
>> Description des besoins fonctionnels

Service de planification des tâches ... Bien mais l'offset est-il nécessaire 

Piste: prendre le meilleur robot jamais réalisé et voir ce dont il a besoin.

~"module de placement" -> banque?
bas niveau
haut niveau
?
os
haut niveau-> code métier

plus haut niveau: Stratégie

Quelle est la frontière entre Simulateur et code dépendant de l'année. Les séparer conceptuellement. PID -> physique, Scheduler -> informatique

Diagramme Simulateur:
  blablabla Ok
  
Et le moteur physique? (ndlr: ahah j'vous l'avait dit)

Ok, Héritage, codeur utilisateur (pas top mais il comprend)

Il préfère les listes.

point obscur: Opérations listées obscur.

Chronogramme sur les broches? Changer la valeur des GPIO?

En fait y a une API entre Aversive++ et le simulateur:

ndlr: Je vous l'avais presque dit!

Idéologiquement l'API est + mieu, mais si elle est pas belle, on la jette! Mais intéressant que si elle oblige à la clarté.

Elle doit répondre au questions "qui parle avec qui?" et "sous quelles contraintes?". Une aide: prototype papier, un putain de schéma trop classe au tableau! et faire vivre le code.

Rôle de l'ancienne API: éviter la régression!

Prendre un peu de temps. Se payer le luxe de prendre du recul. Laisser le temps au bonnes idées d'arriver pour ne pas avoir à tout recommencer.

Loïc a du mal à partir de l'étape 1.

Diagrammes de séquence... de toute beauté, s'appliquer sur les mots.

Prochain rendez-vous: Lundi 17h30
\paragraph{Pour la prochaine fois :}
\subsection*{conclusion}

\end{document}
