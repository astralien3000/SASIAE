\documentclass[a4paper,10pt]{article}

% Options possibles : 10pt, 11pt, 12pt (taille de la fonte)
%                     oneside, twoside (recto simple, recto-verso)
%                     draft, final (stade de développement)

\usepackage[utf8]{inputenc}   % pour les accents
\usepackage[T1]{fontenc}      % Police contenant les caractères français
\usepackage[francais]{babel}  % Placez ici une liste de langues, la
                              % dernière étant la langue principale

\usepackage[a4paper]{geometry}% Réduire les marges
% \pagestyle{headings}        % Pour mettre des entêtes avec les titres
                              % des sections en haut de page

\title{Rapport de réunion}           % Les paramètres du titre : titre, auteur, date
\author{SASIAE}
\date{2 décembre 2013}

\sloppy                      % pour que les lignes ne débordent pas dans les marges

\begin{document}

\maketitle                    % Faire un titre utilisant les données
                              % passées à  \title, \author et \date

\begin{abstract}
Compte rendu de la réunion du mercredi 2 décembre 2013 avec le client Aymeric Vincent.
\end{abstract}
\subsection{Différentes remarques :}
Des remarques qui vont de la formulation d'une phrase pas très claire
: \og L'API \texttt{Aversive++} sera la version 2.0 de l'API
Aversive.\fg , au dessin complet d'un schéma à revoir : Architecture
et message.\\
Lorsqu'on parle de la partie asservissement de l'API aversive++, 
il faut préciser préciser si c'est du projet SASIAE ou bien à la
charge du client.\\
il faut changer \og API \fg par \og implémentation \fg dans API\_AVR
et API\_simulation: ( API = header ) car on parle alors de la
réalisation, et non pas juste de l'interface ( header ).\\
Choisir quel terme employer entre : Plannificateur  et scheduler. De
manière générale il ne faut pas laisser d'ambiguïtés dans le rapport. 
Ce qui signifie qu'un terme employé pour la première fois doit être
défini, et le rapport étant long il est nécessaire de souvent mettre
des références vers les définitions des termes.
\paragraph{Par exemple :} Si je parle de socket pour la première fois,
je définis ce que c'est et ce que c'est dans le cadre du projet, puis
les 5 premières fois où j'en parle je fais référence à cette
définition. \begin{verbatim}\ref{} \end{verbatim}.\\
Page 15, afficher traces récentes d'exécution. On parle de Socket
d'échange, est-ce que c'est bien défini quelque part ? NON le définir
et donner la REFERENCE.
Choisir s'il faut définir les sockets d'échanges dans la partie
communication ou dans la partie Architecture.\\
\paragraph{Schéma :}
Il y a Plusieurs modules ! Il faut le faire apparaître. Il faut revoir
ce schéma et bien donner au coordinateur son rôle central. ( GUI doit être
périphérique)\\
Communication Moteur physique / GUI ? 
\paragraph{Quelques remarques :}
\begin{itemize}
\item ANSI -> ASCII sans accents
\item communication bidirectionnelle : -> on utilisera des sockets {...}
\item Séparé par des saut de ligne "\begin{verbatim}\n\end{verbatim}" lf code ascii 10
\end{itemize}

Faire un diagramme de Gantt et se lancer dans quelques estimations. On
ne doit pas parler \og d'introduction des erreurs \fg, trouver une
autre façon de le dire. Il faut ajouter un identificateur au schéma XSD.
\paragraph{Capteurs : }
Comment fonctionnerait un capteur de contact ? Il est nécessaire de
décrire précisemment le lien qui existe entre les modules ( capteur )
et le moteur physique. \\
PRÉCISEMMENT !!!\\
Savoir exactement comment fonctionne les forces dans le moteur physique.\\
XSD -> Ajouter l'exemple qui a permis décrire le fichier de description dans le rapport
Prévoir la troisième dimension ( z) dans la description d'objet\\
Ajouter dans la référence à Qt une annexe qui explique les slots 
 + paragraphe sur slot Qt pour faire communication proprement.
 

\subsection*{conclusion}
Le rapport d'analyse/cahier des charges fait très professionnel. 
\end{document}
