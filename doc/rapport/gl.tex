\section{Gestion du projet}

Le PFA constitue le projet le plus important que nous ayons réalisé. L'organisation du projet, l'architecture logicielle, la gestion du temps, la répartition des tâches au sein de l'équipe et le découpage de notre équipe en sous-équipes sont autant d'éléments qui ont pris une toute nouvelle envergure pour nous.

\section{Découpage du temps}

Nous avions découpé dans le cahier des charges le projet en plusieurs livrables, allant de la version minimaliste attendue par le client à la version optimale. Le planning a été construit autour de ces cinq livrables de telle sorte que soit réduit le risque de se retrouver sans simulateur utilisable à la fin du projet. Toutes les trois semaines en moyenne un livrable devait être fini pour nous permettre de passer au suivant.
Sur les cinq livrables prévus, nous n'avons pu aller plus loin que le 4ème livrable, mais celui-ci n'en reste pas moins exploitable par le client EIRBOT.

Le planning n'a pas tout à fait respecté les livrables tels quels en fonction de notre avancée. C'est à dire que certaines fonctionnalités ont été intégrées dans un livrable autre que celui prévu originalement, mais les modifications restent minimes. Cela est dû aux différentes difficultés que nous avons pu rencontrer sur des fonctionnalités qui finalement n'étaient pas vitales au projet.

\section{Méthode}

Nous avons principalement travaillé par groupes de deux ou trois pour les différentes tâches, avec des cycles d'environ deux semaines. Le planning de départ a été réparti entre les membres de l'équipe au hasard. Cela a permis à tout le monde de travailler sur tous les domaines abordés par le projet. Le principal problème de cette organisation a été le temps d'adaptation nécessaire à ces changements. De manière à structurer nos travaux nous avons essayé de mettre en place la méthode Kanban et de gérer les tâches en les triant entre les tâches à faire, faites et en cours de développement. Cela n'ayant pas fonctionné longtemps nous avons privilégiés des réunions hebdomadaires de l'équipe et des réunions avec notre encadrant pédagogique toutes les deux ou trois semaines.

\section{Stratégie de tests}

Ce projet a suivi la stratégie de test classique. Pour chaque objet nous avons implémenté des tests unitaires afin de vérifier le respect des spécifications et la correction des méthodes. Puis chaque partie a été intégrée petit à petit afin de vérifier leur bonne articulation et leur respect des attentes. Enfin, nous avons procédé à un test recette en simulant l'exécution d'un robot complet.
Le détail des test unitaires qui ont soulevé des problèmes pertinents est précisé dans la description détaillée du projet.
De plus l'annexe comporte une partie purement descriptive des tests à la section \ref{lestests}. 
L'écriture de test a été bénéfique, en effet la conception des classes et méthodes n'a pas toujours était optimale ou même correcte et les tests ont permis de mettre en évidence ces défauts.
Cependant, au fur et à mesure des itérations de notre méthode de développement certains tests sont devenus obsolètes parce qu'ils étaient liés à un état devenu obsolète de l'implémentation. 


