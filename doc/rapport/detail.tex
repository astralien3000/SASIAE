\label{detail}

Dans cette section seront détaillés les différents aspects / entités qui interviennent dans le simulateur. Nous nous proposons pour chacune des parties de d'abord présenter le contexte de la partie en question, les choix faits dans le cahier des charges, puis de décrire les problèmes et difficultés rencontrées pour finalement opposer ce qui a été fait avec les choix de départ.

\section{Aversive++}

Bien que le développement d'Aversive++ pour microcontrôleur n'entrait pas dans le cadre du PFA, le support des microcontrôleurs AVR utilisés par le client EIRBOT n'en reste pas moins une contrainte forte sur le design de l'API. La deuxième contrainte principale étant son exploitation à des fins de simulations.

\paragraph{} %%\subsection{Choix de conception}

L'une des tâches principales du projet a donc été de ré-implémenter les devices avec lesquels l'association\footnote{EIRBOT est une association loi 1901} travaille. Le travail principal de ces derniers dans le cadre du simulateur étant de traiter les messages reçus de la simulations de manière à reproduire le comportement d'un robot lors d'une exécution de son code. Chaque device doit donc être identifié. Nous avions pensé à les identifier par leur nom dans le cas de ressources internes au microcontrôleur ou par leur pin (port sur lequel est connecté le device) s'ils étaient à l'extérieur du microcontrôleur. Afin de simplifier la communication entre le code robot et le simulateur, un device est systématiquement identifié par son nom. Dans tous les cas, cela a demandé une modification de l'API pour ajouter ce nom en paramètre aux constructeurs des devices (bien qu'utile uniquement à des fins de simulation). À premier abord, cela semble poser problème vis à vis des contraintes d'espace sur AVR. Néanmoins, nos constructeurs de devices étant en \texttt{inline} et le degré d'optimisation de la compilation étant important (\texttt{-O3}), le compilateur ne rajoute pas ces lignes si elles ne sont pas utilisées.

\paragraph{} %%\subsection{Difficultés rencontrées}

Développer Aversive++ a posé de nombreux problèmes, ceux-ci indiqués dans leurs parties respectives détaillées plus bas.

On peut citer néanmoins le type \texttt{SafeInteger} (permettant de faire des opérations arithmétiques \og sûres \fg) que nous n'utilisons finalement pas.

Le client avait demandé un moyen de détecter les erreurs d'overflow lors des calculs. Nous avions donc écrit une classe \texttt{SafeInteger} qui devait être utilisée en lieu et place des entiers classiques. Cependant lors d'une revue de cette classe, nous nous sommes rendus compte qu'elle n'était pas optimisée et ne fonctionnait pas correctement dans certaines situations critiques. Nous avons donc décidé de ne pas l'utiliser. En outre, son utilisation surchargeait les calculs.

\paragraph{} %%\subsection{Produit final}
Le cahier des charges prévoyait déjà un message de synchronisation mais uniquement pour ordonner au code robot de s'exécuter sur un pas de simulation. Ce message a été enrichi avec le temps écoulé depuis le début de la simulation pour permettre au planificateur de tâche de réaliser son travail. Une deuxième information est venu enrichir ce message de synchronisation, il s'agit d'un entier indiquant au code robot combien d'itérations de la boucle principale dans la fonction \texttt{main} il peut exécuter sur ce pas de simulation.

Parmi les contraintes de développement fixées dans le cahier des charges, la plupart on été respectées :

\begin{itemize}
\item utilisation des fonctions \texttt{inline} pour de petites fonctions pour assurer une optimisation du code à la compilation
\item utilisation de variables \texttt{const static} dès que possible
\item utilisation de \texttt{template} à la place de branchement conditionnel si possible de manière à rendre le code générique et plus rapide
\end{itemize}
Nous avions également pensé nous passer de l'utilisation de fonctions virtuelles afin d'alléger l'exécutable pour microcontrôlleur mais cela a posé d'importantes difficultés de conception lors de la création des devices. Cette contrainte est toute fois respectée dans la majorité du code.

\subsection{Client thread}

Aversive++ met à disposition de nombreux devices. Ces devices sont des interfaces haut niveau pour interagir avec du matériel. De ce fait, dans le cas de l'implémentation AVR, les devices apportent une surcouche par dessus les drivers. Hors, dans le cas de la simulation, le matériel n'est pas physiquement présent et il n'y a donc pas de driver non plus. La matériel a donc été représenté par des modules dans le simulateur. Or, ces modules se trouvent dans un processus différent que celui où s'exécutent le code robot et donc les devices. Nous avons donc mis en place un client dans Aversive++ (implémentation SASIAE) pour gérer la communication entre les devices et les modules (le protocole de communication étant détaillé plus loin).  Le client thread permet donc au device \og d'inscrire \fg\ leur nom avec une fonction qui traitera les messages qui leur seront destinés. Il permet également au device l'envoi de message au module ou l'envoi de message de Log. De plus, le \texttt{client thread} s'exécute dans un thread séparé du thread principale (d'où son nom) car il a besoin de continuer à recevoir et interpréter les messages qu'il reçoit, même lorsque le thread principale est bloqué, en attendant que la simulation avance. Une instance de device lui indique le traitement à réaliser lorsqu'il doit lui faire parvenir un message. Mettre en place une telle méthode n'était pas évident dans le cas d'un membre de classe. Nous avons alors décidé de fournir une fonction anonyme, ce qui est possible depuis C++ 2011. Cela permet donc de s'affranchir d'un nouvel héritage d'une classe utilisable seulement par le coordinateur et possédant la méthode de traitement des messages. La mise en place à tout de même demandé un temps d'adaptation à la formulation des fonctions anonymes de C++ 2011.

\paragraph{}
La seule difficulté notable que cette classe a posé était lors de la fermeture d'un exécutable de code robot. En effet, sur la base de la communication entre le \texttt{client thread} et le coordinateur, le \texttt{client thread} faisait une lecture bloquante sur son canal d'entrée et n'arrêtait pas le système correctement. La solution fut de reprendre le principe de fermeture de communication de TCP et de ne s'arrêter que lorsque les deux parties étaient d'accord.

\subsection{Le Scheduler}
Le \texttt{scheduler} transforme le framework Aversive++ en une sorte de système d'exploitation minimal. C'est un élément indispensable pour un robot que de pouvoir traiter plusieurs tâches à la fois et il n'est pas rare que les robots soient équipés d'un système linux embarqué pour traiter leur intelligence. Il ne s'agit pas vraiment d'un ordonnanceur classique, il n'est pas prévus de gérer des threads mais seulement l'enchaînement des tâches. Sans cet élément, un robot ne pourrait pas avoir plus de deux ou trois tâches traités régulièrement. Or, il n'est pas rare qu'un robot ait une dizaine de tâches à traiter quasi-simultanément et régulièrement. 

\paragraph{}
Son principe de fonctionnement est le suivant : toutes les tâches ont une fréquence d'appel et une priorité en cas d'appel simultané. Lors de l'appel à sa fonction de mise à jour, le scheduler doit déterminer quelles taches exécuter et dans quel ordre. Pour faire cela en un temps minimum, il a été choisi de trier les tâches selon leur date de prochain appel (ajoutée à leur priorité) à l'aide d'un tas. La première tache à exécuter est alors à la tête du tas (accès en temps constant). Après traitement d'une tâche périodique, elle est replacée dans le tas avec une priorité adaptée à l'heure de sa prochaine programmation (cela se fait en temps logarithmique en la taille du tas). On suppose que le temps de rafraichissement du \texttt{scheduler} est inférieur à la période de la plus rapide des tâches à traiter, ce qui est vrai en pratique.

Sur simulateur, le comportement est identique. Sauf que sa fréquence de rafraîchissement est donnée par le coordinateur.


\section{Interface graphique}

L'interface graphique est un module particulier du projet. En effet, alors que celle-ci n'a aucune influence sur la simulation en elle-même, elle est tout de même obligatoire pour proposer un outil facilement utilisable par le client.

\paragraph{}

Le framework Qt propose un large panel d'outils pour créer une interface graphique, ainsi que de nombreuses solutions pour gérer des threads, des sockets, des signaux, autant d'objets qui devraient être utilisés par le simulateur. C'est pour cela que nous avons décidé, lors de l'écriture du cahier des charges, d'utiliser ce framework très complet.
L'interface graphique devait pouvoir afficher la table de jeu vue du dessus, en projection 2D, ainsi que le(s) robot(s) se déplaçant dessus et les obstacles du jeu.
Il doit également être possible de charger une table de jeu depuis le système de fichiers, et charger de la même manière un ou plusieurs robots.
De plus, les informations relatives aux capteurs et actionneurs du robot doivent être visibles par l'utilisateur dans une zone dédiée de l'UI.
Une autre zone doit être dédiée à l'affichage de messages d'information et d'alertes provenant du simulateur.
Enfin, il était prévu de donner la possibilité à l'utilisateur de sauvegarder une simulation, l'arrêter, la reprendre, l'accélérer et la ralentir.

\paragraph{}

La première difficulté rencontrée par l'équipe qui devait créer la fenêtre d'affichage fut de se former à l'utilisation de l'outil \texttt{QtDesigner} de \texttt{QtCreator}. Il a fallu ensuite comprendre l'utilisation des signaux Qt. La première version de l'affichage employait le \texttt{Timer} Qt qui permet de gérer la synchronisation de l'affichage. Cela a permis de tester les fonctionnalités d'affichage, le respect des zones, etc... Cependant la simulation n'utilise pas et n'a pas besoin de ce \texttt{Timer}. Elle rythme elle même sa fréquence d'affichage. Il a donc fallut adapter ça pour la suite. Les détails du branchement du calculateur physique à l'interface est décrit plus loin dans la section \ref{calcgui}. Une autre difficulté a été la gestion de la fonction play/pause. Mais ce problème à été corrigé en changeant la position des ordres.

\begin{figure}[!h]
\includegraphics[width=\textwidth]{captureui.png}
\caption{Rendu final de l'interface graphique du simulateur}
\label{captureui}
\end{figure}
\clearpage

\paragraph{}
Finalement, la fenêtre d'affichage ressemble en apparence à celle présentée dans le cahier des charges (voir figure \ref{captureui}). Néanmoins, certaines fonctionnalités ne sont pas disponibles. On ne peut pas enregistrer de simulation, il n'est pas non plus possible d'enregistrer-sous une configuration de simulation. Les fonctionnalités d'insertion d'objets en cours de simulation ne sont pas implémentés.

\section{Calculateur physique}

Le moteur de simulation physique choisi est Bullet. Le cahier des charges indique qu'il est nécessaire pour le bon fonctionnement du simulateur d'utiliser un calculateur physique permettant de simuler les données à envoyer aux capteurs du robot, et de simuler un déplacement du robot proche de la réalité. Le calculateur physique est le coeur de la simulation. Outre le calcul des valeurs à renvoyer aux capteurs qui peut être fait avec une relative exactitude, la simulation d'un déplacement réaliste est le principal enjeu du calculateur physique.

\paragraph{}

L'environnement physique minimal a vu le jour en quelques jours : une table et des murs. Cependant, lors de l'arrivée des premiers tests de déplacement du robot dans cet environnement, le robot restait bloqué sans explication. Ces test reposaient sur des courbes tracées d'après les mesures faites dans le monde simulé. Nous avons donc du revoir notre méthodologie de test. Plutôt que d'afficher seulement les courbes des positions ou des vitesses, nous avons modifié une démonstration fournie avec la bibliothèque pour afficher la simulation comme le montre la figure \ref{capture3d}.

\begin{figure}[!h]
\includegraphics[width=\textwidth]{capture3d.png} 
\caption{Aperçu de la simulation 3D proposée par Bullet}
\label{capture3d}
\end{figure}
\clearpage

Il s'est avéré que les murs n'avaient pas été bien décrits. Il est difficile d'instancier un parallélépipède rectangle dans Bullet et la documentation est réellement ambiguë.


\paragraph{}

Le calculateur physique a finalement été correctement intégré au reste du simulateur, permettant de simuler les valeurs à renvoyer aux capteurs, déplacer le robot sur la table de manière réaliste. Quelques raccourcis ont été effectués, comme par exemple le calcul des valeurs à renvoyer à la balise de détection de robot adverse qui se contente de renvoyer la position relative des autres robots, sans prendre en compte le mécanisme complexe qui permet cette détection.

\paragraph{}

Le choix du moteur physique utilisé pour notre calculateur physique a été fait lors de l'élaboration du cahier des charges et c'est le moteur physique Bullet que nous avons retenu après de rapides tests. Ces tests ont nécessité de nombreuses heures d'installation et de lecture de documentation. Le manque de temps nous a donc empêché de tester d'autres calculateurs physiques. Il aurait été pourtant intéressant d'approfondir l'étude sur Bullet et d'en mener une similaire sur au moins un autre moteur physique. En effet, malgré un forum vivant, de nombreuses démos, la documentation détaillée des classes et méthodes est inexistante. Il existe seulement une documentation générale de haut niveau et une documentation html qui se contente de pointer vers des lignes de code. Le temps à investir pour comprendre l'agencement des différentes objets de Bullet n'a donc pas été négligeable. Dans l'éventualité d'un changement de calculateur nous avons fait en sorte d'interfacer au maximum l'utilisation des fonctionnalités de Bullet. Notamment pour ce qui est des modules.


\section{Modules}

Les modules ont pour but de simuler le comportement d'une partie réelle du robot, que ce soit un capteur ou un actionneur, à l'intérieur du calculateur physique. Ils doivent aussi correspondre à un device du framework et donc pouvoir recevoir des messages de ces derniers. L'objectif principal est de pouvoir imiter facilement les éléments d'origines. Ils interagissent donc directement avec le calculateur afin de lui fournir les informations nécessaires au calcul des pas de simulation et récupèrent les informations qu'ils pourraient récupérer dans la réalité. De plus, l'ajout d'un module doit pouvoir se faire facilement et sans nécessiter la recompilation du simulateur. En effet, EIRBOT évolue et il est important de pouvoir faire évoluer le simulateur en même temps.

\paragraph{}
Pour répondre à cette dernière contrainte nous avons envisagé deux solutions. La première était d'utiliser un langage interprété, le framework Qt permettant entre autres choses d'utiliser des scripts javascript. Le problème de cette méthode est l'interaction avec le reste du simulateur qui reste un peu plus coûteuse que la solution choisie. La seconde solution, et celle retenue est de créer une bibliothèque dynamique. Il n'est donc pas nécessaire de recompiler tout le simulateur lorsqu'on rajoute un module, il suffit de le compiler sous forme de bibliothèque dynamique. Le problème est que l'utilisation de bibliothèques dynamiques en C++ n'est ni simple ni portable. Ces problèmes sont simplifiés par Qt qui fournit de nombreux outils de gestion de bibliothèques dynamiques via les plugins (\texttt{QInterface}, \texttt{QPlugin}, \texttt{QPluginLoader}). Une des difficultés néanmoins rencontrées est que nous avons oublié d'inclure certaines classes à la compilation de la bibliothèque. Or, cette compilation ne lie pas les données et par conséquent ne cherche pas à savoir si les symboles des fonctions utilisées existent vraiment. Par conséquent, ce problème ne s'est révélé qu'à l'utilisation des librairies et à mis un certain temps à être résolu.

\paragraph{}
Au final, les modules ont pu être écrits sans difficultés majeures. L'écriture d'une petite API du côté du calculateur physique permet aux modules de ne pas dépendre du calculateur physique utilisé. Ces dernier remplissent leur rôle en incarnant les devices du framework Aversive++ dans la simulation physique.

\section{Coordinateur}

\paragraph{}
Le coordinateur, comme son nom l'indique, est en quelque sorte le chef d'orchestre du simulateur. Son rôle est d'assurer que chaque partie du projet communique correctement avec les autres, et reste synchronisée. Le diagramme de classe \ref{coordinatorsarchitecture} montre l'architecture que nous avons retenue du Coordinator et de ses dérivés. Il est à noter que celle-ci n'était pas la version que nous avions imaginée lors de la conception de la classe Coordinator. En effet cette classe est restée unique durant un bon mois. Mais le code devenait de plus en plus dense, presque incompréhensible. Nous avons donc choisi de "l'éclater" en utilisant le principe diviser pour régner. Ce processus est détaillé un peu plus loin.

\paragraph{}
Le coordinateur connaît donc tout le système du simulateur mais aussi celui du ou des programmes actuellement testés. Il répertorie chaque module du simulateur et chaque "device" d'Aversive++ et fait en sorte d'assurer la correspondance entre eux. Chaque élément, à son initialisation, doit donc s'identifier auprès du Coordinateur. C'est aussi ce dernier qui récupère les informations du fichier de configuration du robot et crée les éléments nécessaires. Par la suite, le Coordinateur se contente d'aiguiller les signaux Qt entre les différentes entités.

\paragraph{}

\begin{figure}[!h]
\includegraphics[scale=0.5, angle=90]{DiagClasseCoord.png}
\caption{Diagramme de classe de la construction du procédé de coordination}
\label{coordinatorsarchitecture}
\end{figure}
\clearpage

Lors du développement du Coordinateur, nous nous sommes rapidement rendu compte qu'il était surchargé de travail. En effet, il contenait de nombreux objets tels que le \texttt{MainWindow} du rendu graphique, les modules simulant capteurs et actionneurs, les robots simulés, le calculateur physique lui-même, etc. et il devait en plus ordonnancer les calculs !
Dans l'esprit de garder un code plus propre, nous avons donc décidé d'éclater cet unique coordinateur, chacun de ses collaborateurs s'occupant d'une partie spécifique du coordinateur original. Au final le \texttt{Coordinator} construit les autres coordinateurs :
\begin{itemize}
    \item Le \texttt{PhysicalCoordinator} est en charge de la gestion du calcul physique.
    \item Le \texttt{RobotCoordinator} gère la communication avec le code du robot.
    \item Le \texttt{ModuleCoordinator} s'occupe de la gestion des modules.
    \item Le \texttt{GuiCoordinator} est le coordinateur de l'interface graphique. 
    \item Le \texttt{SchedulerCoordinator} est le coordinateur spécialement dédié à l'ordonnancement des tâches et des autres coordinateurs.
\end{itemize}
Cette organisation est récapitulée par le diagramme de séquence \ref{initcoordinators} qui reprend la séquence d'initialisation des coordinateurs, et de connexion des signaux Qt. 

\paragraph{} %%\subsection{Produit final}

\begin{figure}[!h]
\includegraphics[scale=0.4]{DiagSeqInitCoord.png}
\caption{Diagramme de séquence sur l'initialisation des coordinateurs}
\label{initcoordinators}
\end{figure}
\clearpage


\paragraph{} Finalement, malgré la séparation imprévue du coordinateur en différents collaborateurs, le rôle initial du Coordinateur est bien respecté et l'ensemble réalise bien toutes les fonctionnalités prévues par la spécification du cahier des charges. De plus, il peut toujours gérer efficacement la synchronisation des différentes parties du simulateur et aiguiller les messages qui transitent.


\section{Communication}

La communication entre les différentes composantes du projet est primordiale pour le bon fonctionnement du simulateur. Comme il a été vu précédemment dans la partie dédiée au coordinateur, cette communication, qui doit être synchronisée, est complexe. La suite de cette partie consiste à en décrire les différents axes. Avant cela, il est important de noter que toute communication passe par le coordinateur du fait qu'il lie les signaux Qt aux Slots correspondants. La figure \ref{seqopentable} montre un exemple de l'utilisation des signaux Qt. Le principe est simple : supposons qu'il existe 3 classes A, B et C. C possède A et B. A est la classe émettrice qui possède un \texttt{signal} qu'elle \texttt{emit} dans son code. B est la classe réceptrice qui possède un Slot, qui est une méthode public dans cet exemple, avec les même arguments que le signal de A. Enfin il ne reste plus qu'à \texttt{connect(A, SIGNAL(mon_signal()), B, SLOT(mon_slot()))}.  

\paragraph{} 
Une modification importante a été effectuée concernant la communication. Alors que le cahier des charges mentionnait l'utilisation de \texttt{socket} de communication entre le simulateur et le ou les exécutables robots. Il s'est avéré bien plus simple et plus efficace de ne communiquer qu'à partir de manipulation sur l'entrée et sortie standard de ces exécutables robots.

\begin{figure}[!h]
\includegraphics[scale=0.5]{DiagSeqOpenTable.png}
\caption{Diagramme de séquence : choix d'une table par l'utilisateur}
\label{seqopentable}
\end{figure}
\clearpage


\subsection{Calculateur physique et interface graphique}
\label{calcgui}
La communication entre le calculateur physique et l'interface graphique est celle qui permet à l'utilisateur de voir en temps réel l'état de la table. Elle permet à l'interface graphique de rafraîchir régulièrement les informations qu'elle possède à partir des données calculées.

\paragraph{}
Cette partie du projet soulève de nombreuses difficultés. En particuliers, la configuration antique des ordinateurs du client soulèvent des problèmes d'optimisation de la performance, ce qui est un enjeu crucial dans la simulation physique. De manière à fournir un affichage fluide nous avons utilisé le système de signaux de la librairie Qt. Cela résout donc le problème de fréquence d'affichage mais pas celui de l'affichage des robots et autre objets sur la table. La première idée a été de créer une liste pour tous les objets à afficher et de la parcourir à chaque pas de simulation. Mais cette méthode était trop peu modulaire. La solution finale à donc été de créer une classe PMO (\texttt{printable mobile\footnote{La Table est un PMO sans être mobile, mais nous avons estimé que ce détail n'est pas crucial} object}). Cette classe possède les méthodes nécessaires à l'affichage de chaque objet dans la simulation 2D. Tout élément devant être affiché hérite donc de cette classe. La position des objets est ainsi accessible par la méthode \texttt{getPosition} qui utilise directement les fonctionnalités du calculateur physique pour obtenir la position du-dit objet. Donc, plutôt que d'avoir une liste pour chaque type d'objet à afficher il n'y a qu'une liste de PMO à gérer et ses éléments s'y ajoutent d'eux même à leur initialisation.

\paragraph{}
Les PMO ne sont pas des objets Qt et cela nous a posé un problème au cours de l'intégration du projet. En effet Qt crée de nouveaux objets pour générer l'affichage. Il était alors difficile de retrouver quels objets Qt correspondaient à tel ou tel PMO. L'architecture générale a donc dû être modifiée pour relier l'affichage de Qt aux PMO du calculateur physique. Les PMO ont donc un objet \texttt{QGraphicsView}. Enfin le coordinateur rajoute les éléments de la liste des PMO dans l'interface graphique. Afin d'optimiser l'espace mémoire utilisé par le rendu graphique les images (PixMaps) on été rangées dans un ensemble afin d'éviter les doublons et vérifie que l'image n'est plus utilisée avant de la supprimer.

\subsection{Communication du Device au Module}

Un device est un objet (au sens de la Programmation Orientée Objets) qui représente une entrée/sortie du point de vue du code robot. Sur un robot réel, cela peut correspondre à un capteur, un actionneur ou un canal de communication avec un autre système. Dans SASIAE, un device est connecté à un Module, qui se charge de lui envoyer des données, ou de faire des action sur le calculateur physique.
Ainsi, à un module doit correspondre un et un seul device.

\paragraph{}
Les devices et modules sont identifiés par des chaînes de caractères qui représentent leur nom. Ce nom est donc utilisé pour acheminer les messages au bon device ( resp. module). Il est défini par l'utilisateur dans le fichier décrivant le robot pour les modules, et dans le code source à la construction des objets pour les devices.
Un problème peut survenir lorsque plusieurs robots dans la simulation ont des noms de device/module identiques. Il faut donc pouvoir acheminer les messages aux modules et devices qui correspondent au bon robot. Pour cela les robots sont identifiés par un nom représenté par une chaîne de caractères.

Les entités qui s'occupent de l'acheminement des messages sont :
\begin{itemize}
\item Device (Aversive++)
\item ClientThread (Aversive++)
\item RobotCoordinator (SASIAE)
\item ModuleCoordinator (SASIAE)
\item Module (SASIAE)
\end{itemize}
Ces objets se transmettent des messages sous forme de chaîne de caractères. La transmission a été pensée comme un modèle en couches comme le modèle OSI ou le modèle TCP/IP. En effet, à chaque transmission, des informations de routage sont ajoutées au message original pour le transmettre au correspondant.
Le diagramme \ref{commoduledevice} représente le cheminement des messages entre les différentes entités.

\begin{figure}[!h]
\includegraphics[scale=0.5]{DiagComModuleDevice}
\caption{Diagramme présentant le cheminement d'un message d'un device vers un module.}
\label{commoduledevice}
\end{figure}
\clearpage

\paragraph{}

Le protocole de communication entre les modules et les devices est laissé libre (en annexe à la section \ref{convcomm}). Cependant, une convention de messages standards a été définie pour guider à la création de nouveaux modules/devices.
Une fois le message transmis au \texttt{ClientThread} ou au \texttt{ModuleCoordinator}, ce dernier est préfixé par le caractère "D" suivit d'un espace et du nom de module/device. Le \texttt{ClientThread} envoie donc le message au \texttt{RobotCoordinator}.
Le message transmis entre \texttt{RobotCoordinator} et \texttt{ModuleCoordinator} est alors préfixé du nom du robot correspondant.

Un autre problème fut rencontré lorsque plusieurs messages étaient envoyés simultanément. Le coordinateur ne recevait qu'un seul signal alors que plusieurs messages étaient en attente. Le bug fut corrigé en forçant le coordinateur à lire tous les messages disponibles dès la réception d'un signal.

\paragraph{}
Dans le cahier des charges, cet aspect de la communication n'avait pas été abordé avec précision, ce qui explique que de nombreuses modifications ont dû être apportées au fil de l'écriture du code, pour s'adapter aux contraintes qui apparaissaient. Toute fois, ces différences sont transparentes pour les modules et devices qui communiquent tel qu'il était prévu à l'origine.

\section{Langage de description}

Le cahier des charges prévoyait de créer un fichier de description de la table et des robots. Fourni par l'utilisateur il permettra d'adapter le simulateur à la table de l'année et à la configuration interne des robots. La description détaillée des fichiers de configuration est située en annexe du cahier des charges. Ce langage à été prévu en XML afin d'être compréhensible par un utilisateur et directement éditable. Il a donc fallu créer une interface permettant d'extraire les données de ce document afin que le simulateur puisse les exploiter. Et ainsi créer un jeu de structures de données reprenant l'arborescence des fichiers XML. En avançant dans l'intégration des différents éléments nous nous sommes aperçu qu'il manquait certains éléments dans la description du cahier des charges, que nous avons par conséquent rajouté, comme par exemple le poids total du robot, un fichier image pour afficher la table dans le rendu graphique... Pour ce qui est des tests, nous avons simplement testé les différents cas d'utilisation de la fonction d'extraction de données du fichier. Des exemples de fichier de description sont donnés en annexe à la section \ref{descXML}.




