\section{Avancement du projet}

Comme nous l'avons rapidement exposé lors de la description de la stratégie de découpage du temps, nous n'avons pu aller jusqu'au cinquième et dernier livrable. Seul le 4ème livrable a été fini. Cela signifie que nous sommes en mesure de charger des configurations de tables et de robots, de les afficher dans l'interface graphique. La plupart des modules et devices ont été écrits, et seuls ceux qui sont très rarement utilisés par le client EIRBOT ont été mis de côté.
Nous pouvons également rajouter dynamiquement des modules sans recompiler l'ensemble du projet.

\paragraph{}
Très peu de codes robot ont pu être testés jusqu'à présent, mais le client est satisfait de la version produite par l'équipe de PFA.

\paragraph{}
La trajectoire est relativement conforme à celle observée dans la réalité, même si le robot simulé sur-réagit parfois lors de chocs contre le bord de la table. On observe en effet de temps en temps que le robot est à la limite de basculer d'avant en arrière et peut mettre du temps à retrouver son équilibre.

\section{Retour d'expérience}

Ce projet a été très enrichissant. Pour le mener à bout, il a fallu prendre en main des librairies volumineuses, se plonger dans leurs documentations et lutter contre leurs problèmes de compatibilité. Ce projet est aussi le premier véritable projet que nous menons puisqu'il est destiné à un client réel. Le fait que ce projet soit porté par des élèves de l'ENSEIRB-MATMECA ne retire rien à l'importance et à la pression du client puisque nous le côtoyons tous les jours. Finalement, ce projet nous a permis de nous rendre compte de nos capacités à travailler en groupe à des fins pédagogiques et professionnelles. Cela nous a permis de partager nos connaissances et de les appliquer dans des problèmes classiques tels que l'ordonnancement de processus, l'optimisation d'un affichage ou encore la création d'un langage de configuration.

\paragraph{}
L'organisation du travail au sein du groupe a représenté une difficulté importante. Il a fallu gérer les rythmes de chacun. Le travail de groupe est un échafaudage constant basé sur des dissensions, des ententes et des concessions. Un autre problème est de gérer les capacités de chacun. En effet, notre groupe était constitué de membres ayant des maîtrises et des savoirs très différents et certains très avancés. Des discussions se sont souvent retrouvées bloquée plus par entêtement que par réelle difficulté. De plus, il y a eu des difficultés à adapter le rythme du projet au rythme de notre vie scolaire ce qui a donné des périodes de relâchement pénalisantes dans les derniers jours. Comme nous l'avons présenté en début de ce document, nous avons essayé différentes méthodes pour gérer notre projet. Au final, ce sont les réunions hebdomadaires, tous ensemble, qui on été les plus productives. 